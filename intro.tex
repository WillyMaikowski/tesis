\begin{intro}

%Hace alg�n tiempo en la Facultad de Ciencias Fisicas y Matematicas de la Universidad de Chile, en una etapa previa del proceso de titulaci�n se deb�a realizar un recuento manual de los ramos y sus respectivos cr�ditos (Unidades Docentes o UDs). Esto quiere decir que se deb�a corroborar los ramos aprobados con el respectivo plan y t�tulo que se quer�a obtener. Este trabajo que parec�a sencillo, result� ser arduo y complejo, lo que implicaba que una solicitud de recuento demorara meses en ser calculada.

%A partir del a�o 2013 el �rea de Infotecnolog�as (ADI) implement� un sistema de recuentos autom�tico para apoyar esta labor manual, el que implic� una mejora considerable, permitiendo que la misma labor pasara de realizarse de meses a segundos.

%El problema consiste en verificar si un alumno espec�fico cumple con un plan de estudios. Cuando hay m�s de una manera posible de que el plan se cumpla, se busca maximizar la nota con la que deber�a egresar. La tarea de comprobar el cumplimiento del plan actualmente no est� siendo abarcada en su totalidad. El sistema puede calcular que un alumno no cumple con un plan cuando realmente s� lo hace (falso negativo). Lo anterior se considera cr�tico. 

%Tambi�n hay problemas respecto del promedio de titulaci�n que entrega el sistema. Como es posible que un plan de estudios se cumpla a trav�s de distintas combinaciones de cursos y a la forma en que est� implementada la soluci�n, se alcanzan a examinar s�lo algunos resultados posibles, pero no necesariamente se encuentra el �ptimo global. 

%Aunque institucionalmente no es parte del reglamento el que se deba maximizar la nota del avance curricular, esto es de gran inter�s para el alumnado, por lo que es parte de lo que se pretende mejorar.

%Las preocupaciones descritas anteriormente provocan que este recurso, si bien es ampliamente ocupado, se utilice s�lo como informaci�n referencial, debi�ndose revisar posibles errores. Todo lo antes descrito motiv� la propuesta de este tema para mejorar el actual sistema. 

\lipsum[36-40]
\end{intro}
