\begin{conclusion}

A partir de lo ya dicho en la secci�n de ``Validaci�n'', se logr� formular un algoritmo capaz de mejorar la anterior automatizaci�n del recuento de UD's reduciendo considerablemente los tiempos de gran parte de los planes y evidenciar de buena manera el comportamiento que sigue para proximas evaluaciones de los metodos e implementaciones realizadas.

Si bien se encontr� un caso en que el comportamiento no es el �ptimo, se logr� esclarecer las razones constatando c�mo una decisi�n administrativa que afecta la disposici�n de los ramos puede llegar a influir tanto a la descripci�n del avance curricular del alumno como perjudicar el calculo automatizado. 

Como opini�n, se considera que los cambios de la carrera de Ingenier�a Civil Industrial poseen innovaciones en las metodologias o adecuaci�n a las necesidades de los estudiantes, por lo tanto, son medidas atractivas que logran instaurarse dentro del normal desarrollo. Aun as�, debe existir un equilibrio y concenso a nivel de universidad de la instauraci�n de este tipo de medidas en el sentido de que actualmente, como se poseen dominios separados entre departamentos y escuela, se debe velar por el normal funcionamiento local pero a su vez debe existir una cohesi�n para el trabajo en conjunto. Esto incluso implica el asesoramiento que deben tener ciertas implementaciones de normas y reglamentos debido a la falta de visi�n en la posibilidad de implementaci�n que pueden llegar a poseer. Se debe tener cierta certeza de la posibilidad de puesta en marcha antes de firmar un decreto.

Por otra parte, evaluando superficialmente el recuento de UD's en versiones antiguas (por ejemplo en el plan ``Ingenier�a Civil Electricista v2'') se logr� notar ciertos planes que no segu�an un comportamiento logico m�s que el rellenar de alguna manera recuentos que no cumplian todas las reglas (como es el caso de los planes ``Libres''). Esto tambien se debe a que antiguamente el actor principal del proceso tenia la potestad y mayor flexibilidad a la hora de decidir los alumnos con viabilidad de titularse. En cambio, este al ser un sistema automatizado posee mayor rigidez y, si bien en un inicio se poseen planes bien modelados, finalmente se incurre a ``ensuciar'' los planes de estudios a trav�s del tiempo. 

Un ejemplo de esto fue la inclusi�n, hace un tiempo atras, de dos ramos de computaci�n reemplazando el antiguo curso de programaci�n que si bien fue una excelente medida, el modelamiento inicial en el recuento de UD's fue cuestionable y dadas converzaciones termin� ensuciando despreciablemente el plan. Aun as�, recuperando uno de los parrafos mencionados en el ``Marco Te�rico'' acerca de la componente gigante que puede formarse en grafos aleatorios al ir agregando cada vez mas elementos, si bien no es del todo el caso, el ``ensuciar'' de a poco la carrera puede provocar finalmente la generaci�n de una gran componente gigante implicando que parte de las mejoras provocadas en esta memoria debido a la separaci�n de componentes conexas se vea neutralizada.

Por lo anterior, ser�a util la formaci�n de herramientas que permitan guiar y asesorar los lineamientos de los planes de estudios para que si bien permitan innovaciones dentro de �l o la adici�n de elementos creativos, tambien se posea una mirada tecnol�gica con experiencia fundada de la evaluaci�n de dichos planes y vayan analizando a trav�s del tiempo su comportamiento.

\textbf{No se si darle un parrafo a la generalizaci�n que puede darse al recuento en otras facultades como medicina. Generalizar validacion para medi y ver si alcanza el tiempo para agregarlo.}

\end{conclusion}
